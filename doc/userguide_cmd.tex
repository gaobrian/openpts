\lstset{frame=trbl}
\lstset{language=}
\lstset{commentstyle=\textit}
\lstset{basicstyle=\small}

\subsection{ptsc} 

PTS collector.

\begin{lstlisting}[style=help_message]
Usage: ptsc [options] [command]

Commands: (forgrand)
  -i                    Initialize PTS collector
  -t                    Self test (attestation)
  -s                    Startup (selftest + timestamp)
  -u                    Update the RM
  -U                    Update the RM (auto)
  -D                    Display the configuration
  -m                    IF-M mode

Miscellaneous:
  -h                    Show this help message
  -v                    Verbose mode. Multiple -v options increase the verbosity.

Options:
  -c configfile         Set configuration file. defalt is /etc/ptsc.conf
  -P name=value         Set properties.
  -R                    Remove RMs
  -z                    Use the SRK secret to all zeros (20 bytes of zeros)
\end{lstlisting}


\subsection{openpts} 

PTS verifier.

\begin{lstlisting}[style=help_message]
Usage: openpts [options] {-i [-f]|[-v]|-D} <target>
       openpts -D

Commands:
  -i [-f]               Initialize [forcibly] the PTS verifier with the target(collector).
  [-v]                  Verify target(collector) integrity against know measure.
  -D                    Display the configuration (target/ALL)

Miscellaneous:
  -h                    Show this help message
  -V                    Verbose mode. Multiple -V options increase the verbosity.

Options:
  -u                    Selects 'yes' as the the default answer when an update is available [no]
  -l username           ssh username [ssh default]
  -p port               ssh port number [ssh default]
  -c configfile         Set configuration file [~/.openpts/openpts.conf]
\end{lstlisting}

\subsection{uml2dot} 

Generate dot file from the UML State Diagram model.

\begin{lstlisting}[style=help_message]
Usage: uml2dot [options] umlfile

Options
   -o output	        Set output file (default is stdout)

Example:
$ uml2dot -o pcr0.dot pcr0.uml
$ dot -Tpng pcr0.dot -o pcr0.png
$ eog pcr0.png
\end{lstlisting}

\subsection{rm2dot} 

Generate dot file from Reference Manifest (RM).
Select pcr index since the RM may contains multiple FSMs for each PCRs. 

\begin{lstlisting}[style=help_message]
Usage: rm2dot [options] rmfile

Options
	-o output	set output file (default is stdout)
	-p pcrindex	set PCR index
	-l level	set snapshot level (0 or 1)

Example:
$ rm2dot -p 0 -o pcr0.dot rm.uml
$ dot -Tpng pcr0.dot -o pcr0.png
$ eog pcr0.png
\end{lstlisting}

\subsection{iml2text} 

Dump the eventlog in text.
It take out the eventlog from TSS or securityfs file directly.

\begin{lstlisting}[style=help_message]
Usage: iml2text [options]

Options:
  -i filename           Set binary eventlog file (at securityfs)
  -p pcr_index          Select pcr (TSS)
  -I mode               Select IMA's log format (Kernel 2.6.32:32)
  -V                    Verify
  -D                    DRTM
  -E                    Enable endian conversion (BE->LE or LE->BE)
  -h                    Show this help message


Example:
$ iml2text
 Idx PCR       Type    Digest                                EventData
-----------------------------------------------------------------------
   0   0 0x00000008 1dfce7dde0cf13cfff102b1eb01875f752d5090c [BIOS:EV...
   1   0 0x00000001 1c41801dd329198e50a3d98040230095693e49b3 [BIOS:EV...
   2   0 0x00000001 16fb111792cb98a3de12f3abd0406fc04c7e5fca [BIOS:EV...
   3   0 0x00000001 dd261ca7511a7daf9e16cb572318e8e5fbd22963 [BIOS:EV...
<snip>
\end{lstlisting}

\subsection{iml2aide} 

Convert IML to AIDE database.

\begin{lstlisting}[style=help_message]
Usage: iml2aide [option]

Options:
  -c filename           Set config file
  -i filename           Set IMA IML file. default, get IML via TSS
  -r filename           Set AIDE DB file as reference of fullpathname
  -o filename           Set output file (AIDE DB format, gziped)
  -w filename           Set output file (Ignore name list, plain text format)
  -h                    Show this help message

Example:
$ src/iml2aide -c /etc/ptsc.conf -r /var/lib/aide/aide.db.new.gz \
  -o /tmp/aide.db.gz
  AIDE DB(ref) : 241826 entries (/var/lib/aide/aide.db.new.gz)
  IML          : 5681 events (TSS)
  AIDE DB      : 3986 entries (tests/data/Fedora12/aide.db.gz) 
\end{lstlisting}


\subsection{ir2text}

Convert Integrity Report (IR) to text format or binary format (=IML).

\begin{lstlisting}[style=help_message]
Usage: ir2text [options]

Options:
  -i filename           Set IR file
  -o filename           Set output file, else stdout
  -P filename           Set PCR output file (option)
  -b                    Binary, (Convert IR to IML)
  -E                    Enable endian conversion (BE->LE or LE->BE)
  -h                    Show this help message
\end{lstlisting}

\subsection{tboot2iml}

Convert the tboot messages to IML.

\begin{lstlisting}[style=help_message]
Usage: tboot2iml [options]

Options:
  -i filename           txt-stat file to read (default is STDIN)
  -g filename           grub.conf file to read (OPTION)
  -p path               grub path (OPTION)
  -o filename           Output to file (default is STDOUT)
  -v                    Verbose message
  -h                    Help

Example:
  # txt-stat >  txt-stat
  # tboot2iml -i txt-stat >> binary_rtm_measurments

\end{lstlisting}



